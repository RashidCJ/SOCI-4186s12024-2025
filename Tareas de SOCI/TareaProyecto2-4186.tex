\documentclass[11pt]{article}
\usepackage{setspace}    
\onehalfspacing
\usepackage{graphicx}    
\usepackage{amsmath}     
%\usepackage{natbib}      % Apagado en esta tarea
\usepackage{array}       
\usepackage{multirow}    
\usepackage{siunitx}     
\usepackage{textcomp}    
%\usepackage[utf8]{inputenc}  %ignorado, el compilador ya tiene utf8
\usepackage[spanish,es-tabla]{babel}  
\usepackage{csquotes}        
\usepackage[top=1in,right=1in,left=1in,bottom=1in]{geometry} 
\usepackage{makecell}
\usepackage{hyperref}        

\hypersetup{
    unicode=true,            
    pdftoolbar=true,         
    pdfmenubar=true,         
    pdffitwindow=false,      
    pdfstartview={FitH},     
    pdftitle={},             
    pdfauthor={},            
    pdfsubject={},           
    pdfcreator={},           
    pdfproducer={},          
    pdfkeywords={},          
    pdfnewwindow=true,       
    colorlinks=true,         
    linkcolor=blue,          
    citecolor=blue,          
    filecolor=blue,          
    urlcolor=blue            
}

\usepackage[backend=biber,style=apa,natbib=true,language=spanish]{biblatex}
\addbibresource{Referencias.bib}

\defbibenvironment{bibliography}
  {\list{}
    {\setlength{\leftmargin}{0pt}%
     \setlength{\itemindent}{\labelwidth}%
     \addtolength{\itemindent}{\labelsep}%
     \setlength{\itemsep}{\bibitemsep}%
     \setlength{\parsep}{\bibparsep}}}
  {\endlist}
  {\item}

\setlength\bibitemsep{1.5\baselineskip}
\DeclareLanguageMapping{spanish}{spanish-apa}
\title{SOCI 4186-00N\\ Tareas de Proyecto de Investigación \textnumero 2 \\ Reformulación de pregunta y Bibliografía anotada \\ \textsc{Individual o Nombre de Grupo}}
\author{Pon tu nombre aquí \\ Número de estudiante}
\date{\today} %dejen esto como está, con \today

\begin{document}
\singlespacing
\maketitle
Fecha de entrega: 20 de septiembre de 2024.

\begin{enumerate}
    \item Descargarán este documento en el GitHub de la clase y lo subirán a la una carpeta en Overleaf. El compilador a usar es \texttt{XeLaTeX}.
    \item Modificarán la N de SOCI 4186-00N según su sección. Si es individual, dejen \textsc{Individual}; si es grupal, cambien el texto de plantilla por el nombre del grupo.
    \item Cambien el campo \texttt{\textbackslash author\{Pon tu nombre aquí\}}. Si es individual: su nombre completo seguido de \textbackslash\textbackslash y su número de estudiante. Si es en grupo, usen el formato: \texttt{\textbackslash author\{Estudiante 1 \textbackslash\textbackslash \ \textnumero\ de Estudiante 1 \textbackslash and Estudiante 2 \textbackslash\textbackslash \ \textnumero\ de Estudiante 2\textbackslash and Estudiante 3 \textbackslash\textbackslash \ \textnumero\ de Estudiante 3 \textbackslash and Estudiante 4 \textbackslash\textbackslash\ \textnumero\ de Estudiante 4\}}.
    \item Presten atención a la ortografía y sintaxis española (acentos, tildes, diéresis) o escriban en una de las tres modalidades de inglés discutidas: británico, Oxford o estadounidense. Cíñanse a una lengua a través de todo el documento.
%    \item Completarán como mejor sea posible la tarea, y podrán, si entienden necesario hacerlo, trabajar la tarea con compañeres de clase. Si así lo hicieren, pondrán antes de la sección 1 una oración indicándome con quién(es) trabajó la tarea. De lo contrario, dejarán la línea que está actualmente escrita.
    \item Descarguen el PDF tras compilar (icono de flecha descendente) y envíenlo a mi correo electrónico: \href{mailto:rashid.marcano@upr.edu}{rashid.marcano@upr.edu}\footnote{A menos que haya habilitado para ese momento un formato de entrega vía Moodle.} con el Asunto (en inglés, \textit{Subject}) `Tarea Proyecto 2 - SOCI 4186-001' si es de la sección 1 (lunes y miércoles) o `Tarea Proyecto 2 - SOCI 4186-002' si es de la sección 2 (martes y jueves).
\end{enumerate}

\section*{Formato del documento}
Siga esta plantilla, borrando las instrucciones si así deseare tras concluir las respuestas o tener el PDF de instrucciones a la mano. El documento debe incluir:
\begin{itemize}
    \item Reformulación de la pregunta de investigación.
    \item Citas y anotaciones de cada fuente. Las fuentes deben ser artículos académicos o libros publicados por editoriales universitarias. No incluya informes de think tanks.
\end{itemize}
\onehalfspacing
\section*{Parte 1: Reformulación de la pregunta de investigación}

Refine su pregunta inicial de investigación en una versión de 250-500 palabras, asegurándose de responder en ese espacio a las siguientes preguntas:
\begin{itemize}
    \item ¿Es inductiva o deductiva?
    \item ¿Por qué es interesante e importante? (Revise la diapositiva 10 de la presentación \href{https://github.com/RashidCJ/SOCI-4186/blob/main/Gu%C3%ADas%20y%20manuales%204186/Repaso%20de%20investigaci%C3%B3n%20en%20ciencias%20sociales.pdf}{Repaso de investigación en ciencias sociales} para recordar cómo argumentar la relevancia o importancia).
    \item ¿Cómo aborda una brecha en la literatura relevante?
    \item ¿Cuáles son las implicaciones de responderla? (¿Qué aprenderemos de su investigación?).
\end{itemize}

\section*{Parte 2: Bibliografía anotada}

Una vez que haya delimitado su tema, elabore una mini bibliografía anotada que incluya entre cinco y ocho fuentes académicas relacionadas con su tema. Cada fuente debe incluir una cita y un breve resumen (un párrafo) de los puntos principales del artículo. Las fuentes académicas incluyen artículos en revistas académicas y/o libros publicados por editoriales académicas (normalmente, estos tienen la palabra «Universidad» en su nombre). Las fuentes académicas NO incluyen informes de \textit{think tanks}, ya que dichos informes no son revisados por pares; tampoco incluyen artículos de periódicos o revistas no académicas. Sus fuentes para la bibliografía anotada deben ayudarle a desarrollar sus conceptos, medidas y marco teórico/analítico. Estas fuentes NO deben ser fuentes de datos.

Elabore una mini bibliografía anotada de 5 a 8 \textbf{fuentes académicas} relacionadas con su tema. Incluya:
\begin{itemize}
    \item Una cita por fuente (el formato de cita \texttt{\textbackslash bibliographystyle\{apalike\}} ya está en el preámbulo para el formato).
    \item Un párrafo resumen (tres o cuatro oraciones) por fuente, explicando los puntos principales y cómo esta es útil para su investigación. No copie textos directamente de las fuentes. Las reglas de plagio aplican.
\end{itemize}

Ejemplo de una entrada en la bibliografía anotada.

\begin{enumerate}
  \item \citet{hall2009institutional}. \emph{Institutional change in Varieties of Capitalism}. Socio-Economic Review, 7(1), 7--34.

  \textbf{Anotación:} En este artículo, Hall y Thelen exploran cómo las instituciones económicas evolucionan en diferentes sistemas de economía de mercado. Analizan los mecanismos de cambio institucional y argumentan que las instituciones no son estáticas, sino que cambian a través de procesos graduales influenciados por actores políticos y económicos. El estudio adopta un enfoque deductivo, desarrollando su marco teórico a partir de conceptos existentes, y utiliza análisis teórico y comparativo para determinar sus hallazgos estudiando casos.
\end{enumerate}

%\bibliography{Referencias} % Archivo de bibliografía para natbib, ya no es necesario.
%\newpage %sólo necesario si quiere asegurar que la bibliografía esté en un sitio más aceptable estéticamente.
\printbibliography[title={Bibliografía}, heading=subbibliography]

\end{document}