\documentclass[11pt]{article}
\usepackage{setspace}    
\onehalfspacing
\usepackage{graphicx}    
\usepackage{amsmath}     
\usepackage{natbib}      % Para manejar las citas y referencias
\usepackage{array}       
\usepackage{multirow}    
\usepackage{siunitx}     
\usepackage{textcomp}    
\usepackage[utf8]{inputenc}  
\usepackage[spanish,es-tabla]{babel}  
\usepackage{csquotes}        
\usepackage[top=1in,right=1in,left=1in,bottom=1in]{geometry} 
\usepackage{makecell}
\usepackage{hyperref}        

\hypersetup{
    unicode=true,            
    pdftoolbar=true,         
    pdfmenubar=true,         
    pdffitwindow=false,      
    pdfstartview={FitH},     
    pdftitle={},             
    pdfauthor={},            
    pdfsubject={},           
    pdfcreator={},           
    pdfproducer={},          
    pdfkeywords={},          
    pdfnewwindow=true,       
    colorlinks=true,         
    linkcolor=blue,          
    citecolor=blue,          
    filecolor=blue,          
    urlcolor=blue            
}

\bibpunct{(}{)}{;}{a}{}{,}  % Configuración de estilo de cita
\bibliographystyle{apalike} % Estilo de cita apalike

\title{SOCI 4186-00N\\ Tareas de Proyecto de Investigación \textnumero 1 \\ Pregunta inicial de investigación}
\author{Pon tu nombre aquí}
\date{\today}

\begin{document}
\singlespacing
\maketitle
\onehalfspacing
Fecha límite de entrega: 30 de agosto de 2024.

\begin{enumerate}
    \item Descargarán este documento en el GitHub de la clase, y lo subirán a la misma carpeta que abrieron cuando hicieron cuenta conmigo en Overleaf el lunes (probablemente llamada Proyecto).
    \item Cambiarán el nombre que sale arriba (al final del preámbulo, poco antes del \texttt{\textbackslash begin\{document\}}, donde dice \texttt{\textbackslash author\{Pon tu nombre aquí\}} y pondrán en su lugar su nombre(s) y apellidos.
    \item Prestarán atención a la ortografía de la lengua española, asegurándose de colocar acentos, diéresis, o tildes donde deban ir. Para esto, tienen dos opciones 
    \begin{enumerate}
        \item si su teclado estuviera en español, usen los símbolos donde es menester,
        \item si su teclado no estuviera en español, está la opción indicada en clase de usar \texttt{\textbackslash} y el símbolo necesario (por ejemplo, ' para acento (\'a o \'{a}), "\ para diéresis (\"u o \"{u}), \~\ para \~n)
        \item Alternativamente, si prefiriera escribir en inglés \underline{la tarea entera}, podrá hacerlo, ciñéndose estrictamente a una de las tres ortografías principales de ese idioma, es decir, inglés británico, inglés Oxford, o inglés estadounidense.
    \end{enumerate}
    \item Completarán como mejor sea posible la tarea, y podrán, si entienden necesario hacerlo, trabajar la tarea con compañeres de clase. Si así lo hicieren, pondrán antes de la sección 1 una oración indicándome con quién(es) trabajó la tarea. De lo contrario, dejarán la línea que está actualmente escrita.
    \item  Una vez terminen, apretarán el símbolo de descarga (a la derecha del botón de Compilar, con flecha descendiente), y descargarán el PDF para enviarlo a mi persona.
    \item Someterán la tarea a mi correo electrónico: \href{mailto:rmarcano@iu.edu}{rmarcano@iu.edu} con el Asunto (en inglés, \textit{Subject}) 'Tarea de SOCI 4186-001' si es de la sección 1 (lunes y miércoles) o 'Tarea de SOCI 4186-002' si es de la sección 2 (martes y jueves).
\end{enumerate}

\section*{Pregunta Inicial de Investigación y Modalidad}
Les estudiantes identificarán un tema de interés, realizarán una breve búsqueda sobre su tema para que puedan empezar a concretar de tema de interés a pregunta. Deberán entregar un documento de al menos 300 palabras que incluya los siguientes componentes (por favor, etiqueta cada componente como una subsección):
\begin{enumerate}
    \item Su tema de investigación principal.
    \item Una pregunta de investigación tentativa, o varias preguntas si no estuvieran segures de qué dirección tomar.
    \item Una breve explicación de por qué desean explorar la pregunta/tema y por qué merece un análisis académico.
    \item Indicación de modalidad de estudio: grupal o individual. Si fuera grupal, información de miembros que conformen el grupo y nombre para el grupo.
\end{enumerate}


\section*{Ejemplo}
Mi tema de investigación es \textbf{tal y cual}. Mi pregunta tentativa de investigación es ¿Cómo tal y cuál se relacionan en el contexto de Jurutungo?

La investigación en ciencias sociales requiere un enfoque riguroso y bien definido. Según \citet{Perez2021}, es esencial establecer una pregunta de investigación clara para guiar el estudio sobre \textbf{tal y cual}. Además, \citet{Gonzalez2022} menciona la importancia de delimitar el área de estudio para enfocar los esfuerzos de investigación, en especial cuando trabajamos con temas provenientes de Jurutungo. Y el clásico de Weber indicaba que \textbf{tal} y otra cosa es importante \citep{Weber1964} para entendernos en sociedad. Por eso entendemos que este tema de estudio merece ser estudiado utilizando métodos cuantitativos o computacionales varios como X Y ó Z.

Este estudio estará siendo de carácter (individual/grupal).

\section*{Miembros del Grupo}
Si el trabajo se realiza en grupo, los miembros del grupo deberán ser declarados en la tabla siguiente. Denle un nombre al grupo en este caso. El límite es de cuatro miembros. De no ser en grupo, meramente declarará el estudiante que estará trabajando en esto por su parte.

\begin{table}[h!]
    \centering
    \begin{tabular}{|c|c|c|}
        \hline
        \textbf{Nombre de Estudiante} & \textbf{Concentración} & \textbf{\textnumero\ de Estudiante}\\ \hline
        Miembro 1 & &\\ \hline
        Miembro 2 & &\\ \hline
        Miembro 3 & &\\ \hline
        Miembro 4 & &\\ \hline
    \end{tabular}
    \caption{Miembros del grupo `los Guaraguaos'}
    \label{tabla:miembros_grupo}
\end{table}

\newpage
\bibliography{Referencia} % Archivo de bibliografía

\end{document}